% ■ アブストラクトの出力 ■
%	◆書式:
%		begin{jabstract}〜end{jabstract}	:日本語のアブストラクト
%		begin{eabstract}〜end{eabstract}	:英語のアブストラクト
%		※ 不要ならばコマンドごと消せば出力されない.



% 日本語のアブストラクト
\begin{jabstract}
テンプレートの説明を,テンプレート自身を使って説明する.これは @kurokobo による卒業論文のための\LaTeX テンプレートを修士論文用に改造し,さらにUTF-8化やMakefile等の添付をしたものである.

この部分には一般には論文のアブストラクトを書く.日本語のアブストラクトを書きたいなら,\verb|\begin{jabstract}| と \verb|\end{jabstract}| の間に文章を書けば,今のこのページのように体裁が勝手に整って出力される.英語のアブストラクトは \verb|\begin{eabstract}| と \verb|\end{eabstract}| の間に書けば,次ページのような体裁で出力される.

両方を書けば,日本語と英語の両方のアブストラクトが並んで出力される(この文書はサンブルなので両方書いてある).ページ順序は,コマンドを書いた順序の通り.どちらか一方のみを出力したい場合は,不要な方をコマンド自体を含め削除する.

このあたりの詳細もあとで書く.基本的には,\texttt{main.tex}を上から順にいじっていけばできるはず.
\end{jabstract}



% 英語のアブストラクト
\begin{eabstract}

\sassan{本項目は, ダミーテキストに置き換えてあります. 詳しくは, texファイルのコメント参照. }

% ダミーテキストとして,lipsumを挿入.
% \lipsum[1-2]とか書けば,2パラグラフ挿入できます.
\lipsum[1]

\end{eabstract}
